\chapter{Terrain LOD in Real-world Systems}
\section{Game Engines}
\subsection{Godot}
Godot is a cross-platform game engine written in C\#, C++ and its own scripting language GDScript.
Terrains are supported in form of extensions developed by community members, 
which can be installed and used in Godot projects by game developers.

One such extension is Terrain3D by Cory Petkovsek \cite{godotterrain3dgithub} 
written in C++ for Godot 4. The LOD approach used in this extension is based on 
geometry clipmaps by Hoppe and Losasso \cite{geomclipmaps}. % TODO Explain algo in high-level fashion
The concrete implementation
of the geometry clipmap mesh code was created by Mike J Savage \cite{geomclipmapssavage}. 

% TODO https://github.com/Zylann/godot_heightmap_plugin/blob/11e0db92c2f8e0e907ae173b983d73af1e35c4c2/addons/zylann.hterrain/doc/docs/index.md
Another extension for terrains is Godot Heightmap Plugin by Marc Gilleron \cite{godotheightmapplugingithub} written in GDScript and C++.
The extension uses a quadtree-based approach for terrain LOD. 

\subsection{Unity}
Unity is another cross-platform game engine written in C\# and C++, and has a 
built-in terrain system. The core engine source code of Unity is only accessible
by owning an enterprise licence, therefore no information is given on which 
terrain LOD is used for the built-in terrain in Unity.

Nonetheless, there exists an open-source library for hierarchical LOD in Unity called HLODSystem developed by JangKyu Seo at Unity TODO citation .
HLODSystem also supports terrains with its TerrainHLOD component, allowing for conversion from an Unity Terrain object to a HLOD mesh with configurable parameters, such as chunk size and border vertex count.
HLODSystem allows the developer to specify the mesh simplifier to be used and currently the only supported simplifier is UnityMeshSimplifier, 
an open-source mesh simplifier developed by TODO that utilizes the fast quadratic mesh simplification algorithm developed by TODO citation.

In addition to the mentioned open-source libraries, a terrain LOD approach was published by Jonathan Dupuy at Unity.

\subsection{Unreal Engine}
Unreal Engine is another cross-platform game engine written in C++ and features
an integrated terrain system called the Landscape system.
The technical documentation of Unreal Engine 5 mentions utilising GeoMipMapping for 
handling LOD for landscapes \cite{unrealengine5doc}. GeoMipMapping is a terrain LOD approach developed 
in 2000 by de Boer \cite{geomipmapping}. % TODO Explain algo in high-level fashion

\subsection{Frostbite}
Frostbite is a closed-source game engine developed by DICE and is known for the \textit{Battlefield} series.
During the Game Developers Conference 2012, DICE presented the terrain system of \textit{Battlefield 3}, which was developed with their Frostbite 2 engine.
They mention a quadtree-based terrain LOD system and describe several optimizations regarding paging and streaming of terrain.
TODO cite.

\section{Geographic Information Systems}
TODO: Check out Google maps, Google Earth, other GIS
TODO: Maybe look at flight simulators (if information available)
