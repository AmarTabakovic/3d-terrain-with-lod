\chapter{Existing Work and Literature}
\section{Research Articles and Publications}
Terrain LOD is a well-researched topic and over the last three decades, numerous approaches have been published.
In the following, some of the most important publications are listed in chronological order.
The approaches that are described in more detail in chapter ``Algorithms and Approaches for Terrain LOD'' are highlighted in \textbf{bold}:
\begin{itemize}
  \item ``Real-Time, Continuous Level of Detail Rendering of Height Fields'' \cite{lindstrom1996} by Lindstrom \textit{et al.} in 1996.
  \item \textbf{``ROAMing Terrain: Real-time Optimally Adapting Meshes''} \cite{roam} by Duchaineau \textit{et al.} in 1997.
  \item ``Real-Time Generation of Continuous Levels of Detail for Height Fields'' \cite{rottgerpaper} by Röttger \textit{et al.} in 1998.
  \item \textbf{``Fast Terrain Rendering Using Geometrical MipMapping''}  \cite{geomipmapping} by de Boer in 2000.
  \item ``Rendering Massive Terrains using Chunked Level of Detail Control'' \cite{chunkedlod} by Ulrich in 2002.
  \item ``Geometry Clipmaps: Terrain Rendering Using Nested Regular Grids'' \cite{geomclipmaps} by Hoppe and Losasso in 2004.
  \item ``Continuous Distance-Dependent Level of Detail for Rendering Heightmaps (CDLOD)'' \cite{cdlod} by Strugar in 2009.
  \item ``Concurrent Binary Trees (with application to longest edge bisection)'' TODO cite by Dupuy in 2020.
\end{itemize}

\section{Level of Detail for Computer Graphics}
\textit{Level of Detail for Computer Graphics} by Luebke \textit{et al.} \cite{lodfor3dgraphics} 
is a reference book for the topic of 
LOD published in 2002. The book builds on top of years of research in the 
area of LOD and provides an overview to many LOD techniques. For this project,
chapter 7 ``Terrain Level of Detail'' of the book is especially relevant, 
as it dives into the topic of LOD for terrains specifically.
It covers basic approaches and techniques for terrain LOD, 
common problems that can arise during rendering of terrains and some solutions to them, 
and a catalog of terrain LOD algorithms.

\section{Focus on 3D Terrain Programing}
\textit{Focus on 3D Terrain Programming} by Trent Polack \cite{focuson3dterrainprogramming} is a book on terrain programming published in 2002.
Part one of the book introduces the reader to the basics of terrains, such as height maps and texturing. Part two then 
covers some more advanced topics, including a selection of terrain LOD algorithms. The presented LOD algorithms are
ROAM \cite{roam}, Röttger's quadtree-based algorithm \cite{rottgerpaper}, and GeoMipMapping \cite{geomipmapping}.
The book also includes demo source code in C++ and OpenGL.

\section{Virtual Terrain Project}
The \textit{Virtual Terrain Project} \cite{vtp} was a project run from 2001 to 2013 
that consisted of a collection of software, information and resources on terrain modelling and rendering,
including a large overview of publications and implementations related to terrain LOD algorithms.
