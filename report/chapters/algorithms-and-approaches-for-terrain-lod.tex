\chapter{Algorithms and Approaches for Terrain LOD}
\section{Basics of Terrain LOD}
\subsection{Terrain Data Representation}
One way of representing terrains is using \textit{heightmaps}.
A heightmap is a $n\times n$-grid that contains 
the height value $y$ for each $(x,z)$-position.
Positions are always spaced evenly in a grid-like manner,
but the distance between two neighboring vertices is variable
and can be set by the terrain programmer.

The main advantages of heightmaps are
easy storage and manipulation of height values. %The terrain data can simply be read into the graphics API as vertices $(x,y,z)$ defined by the current grid position $(x,z)$ and height value $y$.
Terrain height data can be easily stored as grayscale images,
where low grayscale values represent low areas of terrain and vice versa for
high grayscale values. Figure 1 shows a $2000 \times 2000$ heightmap of the mountain Dom in Valais, Switzerland.
\begin{figure}[H]
  \centering
  \includegraphics[width=0.44\textwidth]{dom}
  \caption{$2000 \times 2000$ heightmap of the mountain Dom in Valais, Switzerland retrieved from SwissTopo \cite{alti3d}.}
\end{figure}

An alternative to the heightmap is the \textit{triangulated irregular network (TIN)} data structure.
A TIN consists of a collection of 3D vertices where 
the arrangement of vertices can be irregular.
The main advantage of TINs is that fewer polygons need to be used for 
e.g. smooth terrain areas. However, view culling, deformations and terrain following are 
more difficult to implement. Also, the full $(x,y,z)$ coordinates need to be stored,
whereas with heightmaps, only the height value $y$ needs to be stored.

\subsection{Bintrees and Quadtrees}
Binary triangle trees (bintrees) and quadtrees are 
recursive data structures based on triangles and quads respectively.

\subsection{Potential Problems}
% TODO: Mention cracks, T-junctions, etc.

\section{ROAM}
ROAM (short for \textbf{R}eal-time \textbf{O}ptimally \textbf{A}dapting \textbf{M}eshes) is an algorithm by Duchaineau \textit{et al.} \cite{roam} published in 1997.

ROAM represents the terrain mesh using bintrees. 

The algorithm is driven by two priority queues: a split queue and a merge queue.
The split queue contains triangle split operations 

The main advantage of this approach is that the terrain mesh from a previous 
frame can be used for the current frame. The mesh doesn't have 
to be built from ground up each frame.


\section{CDLOD}


