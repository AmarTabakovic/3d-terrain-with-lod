\chapter{Algorithms and Approaches for Terrain LOD}
\section{ROAM}
\textit{ROAM} (short for \textbf{R}eal-time \textbf{O}ptimally \textbf{A}dapting \textbf{M}eshes) 
is a terrain LOD algorithm developed by Duchaineau \textit{et al.} \cite{roam} published in 1997.
ROAM represents the terrain mesh using bintrees and performs triangle splits and merges
for generating and removing detail. 

\subsection{General Idea}
The central idea of the algorithm is to use temporal coherence: the mesh from a previous frame is used to compute 
the mesh of the current frame, rather than building up the mesh from ground up for each frame.
This is done using two priority queues: a split queue $\mathcal{Q}_s$ and a merge queue $\mathcal{Q}_m$.
The split queue contains splittable triangles $T$
and the merge queue contains mergable triangle pairs $(T,T_B)$.
The elements of the priority queues are ordered by 
various geometric error metrics, which are explained in the subsection ``Error Metrics'' of this report.
At each frame, the terrain mesh gets split and merged using $\mathcal{Q}_s$ and $\mathcal{Q}_m$. until either the required size/accuracy is reached
or the time runs out.
ROAM is designed as a greedy algorithm, meaning it will always performs the most optimal splits/merges for each frame.

\subsection{Error Metrics}
The original ROAM paper mentions various error metrics for the priority queue ordering,
which are explained in the following paragraphs.

\paragraph{Wedgies} \textit{Wedgies} are nested bounding volumes around triangles that are computed 
while building the initial mesh at the beginning of the algorithm.
A wedgie is defined to contain the entire $x$ and $z$ extent\footnote{The original ROAM paper uses $z$ for the up direction. For the sake of consistency with the rest of this report, we will use $y$ as the up direction here.}
of a triangle and the height $y$ including some additional padding space above and below the highest and lowest points,
respectively.

\paragraph{Geometric Screen Distortion}
Another metric is the distance between where a node is supposed to be on the screen and where the algorithm placed the node.
The maximum of all distances is calculated and used as the base priority metric of the algorithm.

\subsection{Other Optimizations}
\paragraph{View-frustum Culling}
An optimization that is mentioned is view-frustum culling. 
View-frustum culling is done on a per-bintree-triangle.
In each frame, various flags are updated during 
the recursive traversal of the bintree. These flags indicate whether a wedgie is inside the view-frustum 
fully, partially, or not at all.

\paragraph{Incremental Triangle Stripping}
The ROAM paper reports using \textit{incremental triangle stripping} for optimizing the performance of the rendering.
This is simply refers to using triangle strips for rendering. which is supported by all current major graphics APIs.

\subsection{Conclusion}
ROAM is not particularly suited for today's GPU, since it mainly relies 
on immediate mode rendering, which is outdated in most graphics APIs of today.
In addition to this, the splits and merges of the priority queues happen entirely on the CPU.

%\section{Röttger's Quadtree-based Algorithm}
%TODO

\section{GeoMipMapping}
\textit{Geometrical Mipmapping (GeoMipMapping)} is a terrain LOD approach developed by de Boer \cite{geomipmapping} in the year 2000. 
It applies the idea of texture mipmapping to terrain rendering. In this section, all 
presented ideas are from the original paper, unless noted otherwise.

\subsection{General Idea}
The central idea of GeoMipMapping is its analogy to texture mipmapping: just like how textures of far away objects are rendered using lower resolution texture mipmaps,
terrain areas that are far away from the camera should also be rendered with a lower resolution mesh.

This is achieved by splitting up the terrain into so-called \textit{blocks} (also called \textit{patches}) of a fixed width $2^n + 1$ for an arbitrary $n \in \mathbb{N}$.
Each block has a LOD level\footnote{The original GeoMipMapping paper uses 0 to denote the maximum LOD level and vice-versa for the minimum LOD level. In order to avoid any confusion with the term \textit{LOD}, this report denotes 0 as the minimum LOD level and vice versa for the maximum LOD level.} $0\leq l \leq n$ that changes dynamically at runtime.
At load time, every block is computed at every LOD level and subsequently stored, e.g. on an index buffer.
Each such representation of a block at a specific LOD level is called a \textit{GeoMipMap}.
For each GeoMipMap, the number of vertices on one side is $2^{l}+1$ and the number of quads is $2^{2l}$.

For example, the GeoMipMaps of a $5 \times 5$ terrain block contain $2^2 + 1 = 5$ vertices on one side at the maximum LOD level 2, $2^1 + 1 = 3$ vertices at LOD level 1 and $2^0 + 1 = 2$ vertices at the minimum LOD level 0.
As for the number of quads, at LOD level 2 there are $2^{4} = 16$ quads, at LOD level 1 there are $2^2 = 4$ quads and at LOD level 0 there is $2^0 = 1$ quad.
Figure~\ref{fig:geomipmapping-patch-example} illustrates the above example.

\begin{figure}[H]
  \centering
  \subfloat[\centering LOD level 2 (maximum).]{{\includegraphics[width=0.28\textwidth]{geomipmapping-level-2.png} }}
  \qquad
  \subfloat[\centering LOD level 1.]{{\includegraphics[width=0.28\textwidth]{geomipmapping-level-1.png} }}
  \qquad
  \subfloat[\centering LOD level 0 (minimum).]{{\includegraphics[width=0.28\textwidth]{geomipmapping-level-0.png} }}
  \caption{Example of each GeoMipMap of a $5 \times 5$ block. The omitted vertices of lower LOD GeoMipMaps are marked as dotted circles.}\label{fig:geomipmapping-patch-example}
\end{figure}

\subsection{View-frustum Culling}
The organisation of the terrain into blocks allows for easy view-frustum culling.

\subsection{LOD Selection}
The LOD for each block is selected at runtime and is based on a 

\subsection{Avoiding Cracks}
GeoMipMapping avoids cracks by checking whether the
current block has a higher LOD than the bordering block and if this is the case,
it omits the vertices that would cause the crack.
The vertex omission is performed by rendering the bordering row/column of the current block
as triangle fans, as shown in figure~\ref{fig:geomipmapping-crack-avoidance}.

\begin{figure}[H]
  \centering
  \includegraphics[width=0.7\textwidth]{geomipmapping-crack-avoidance}
  \caption{Example of GeoMipMapping's crack avoidance between a LOD 2 and a LOD 1 GeoMipMap of two $5 \times 5$ blocks.}\label{fig:geomipmapping-crack-avoidance}
\end{figure}

Regardless of the rendering approach, a neighborhood structure consisting of the LOD levels of the left, right, top and bottom 
neighboring blocks needs to be stored per block, so that the algorithm knows how to perform the vertex omission. The LOD levels of each block change each frame, which means 
that the neighborhood structure of each block also gets updated each frame.

\subsection{Other Optimizations}
\paragraph{Vertex Morphing}
GeoMipMapping can be extended with vertex morphing in order to decrease popping.

\section{(GPU-based) Geometry Clipmaps}
Geometry Clipmaps is a terrain rendering technique published by Hoppe and Losasso in 2004.
A follow-up GPU-based variant of Geometry Clipmaps was described in GPU Gems TODO by Hoppe and Asirvatham in 2005.
In this section, the basic features of the GPU-based Geometry Clipmaps algorithm are presented.

The general idea is to have a single flat mesh centered around the viewer.
The flat mesh consists of nested rings which double in size each level.
The height values are stored in a texture object and the flat mesh is displaced in the vertex shader.

\subsection{Mesh Organisation}
The algorithm is based on a single flat mesh centered around the camera.
The flat mesh is organized as a set of nested rings of $l$ levels, 
where the innermost level $l-1$ is filled in, and where the ring at level $l_i$ is twice as big as the ring 
at level $l_{i+1}$. 

\subsection{Vertex Morphing}

\subsection{View-frustum Culling}

%\section{CDLOD}
%TODO

\section{Concurrent Binary Trees}
The approach is similar to ROAM in the sense that it computes a triangulation of the terrain using bintrees.
The main difference is that the spliting and merging of the bintrees happens in parallel on compute shaders using the 
CBT data structure, whereas in ROAM, the bintrees are split and merged on the CPU.

\subsection{Conclusion}
The paper does not mention any methods used to prevent popping of the terrain.

