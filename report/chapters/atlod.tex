\chapter{ATLOD: A Terrain Level of Detail (Renderer)}
This chapter describes \textit{ATLOD} (short for \textbf{A} \textbf{T}errain \textbf{L}evel \textbf{o}f \textbf{D}etail (Renderer)), the demo terrain rendering application.

\section{Preliminaries}
\subsection{Used Technologies}
ATLOD is written in C++17 and OpenGL 4.2.
For compiling build files, CMake (minimum version 3.5) is used.
ATLOD uses the following third-party libraries:
\begin{itemize}
  \item GLM: The \textit{OpenGL Mathematics (GLM)} library provides functionality for the mathematics of graphics programming, such as classes for vectors, matrices and perspective transformations.
  \item GLEW: The \textit{OpenGL Extension Wranger Library (GLEW)} is an extension loading library for OpenGL. 
  \item GLFW: 
  \item STB: STB is a collection of header-only libraries developed by Sean Barrett TODO cite. ATLOD uses \texttt{stb\_image.h} for loading images of heightmaps and textures.
\end{itemize}

The source code is hosted on GitHub on the repository AmarTabakovic/3d-terrain-with-lod
and is licensed under TODO.

\subsection{Chosen Algorithms}
The chosen algorithms and their reasons for implementing them are the following:
\paragraph{Naive brute-force algorithm} The naive brute-force algorithm consists of simply reading in all height values from the heightmap as vertices and rendering them directly to the screen.
The main reason for implementing the naive brute-force algorithm is to motivate the usage of terrain LOD algorithms by showing the difference in performance
compared to the optimized LOD approaches.

% TBD, depends on how far I come
\paragraph{De Boer's GeoMipMapping} The GeoMipMapping
% \paragraph{Hoppe and Lossaso's Geometry Clipmaps} TODO
% \paragraph{Röttger's Quadtree-based Algorithm} TODO
% \paragraph{Strugar's CDLOD} TODO

\subsubsection{Algorithms Not Chosen}
\paragraph{ROAM} Duchaineau's ROAM was not chosen to be implemented due to its CPU-heavy nature, which makes it not suitable
for today's highly performant GPUs.

\section{Architecture}
TODO High-level class diagram
