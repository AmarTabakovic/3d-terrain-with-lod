\chapter{Discussion}
Overall, the implemented algorithm works decently well, despite 
lacking some features for it to be fully optimized.
The configurability of the implementation allows for usage of the 
system for different applications and purposes.

In order to actually test the limits of the implemented algorithm, the performance measurements
should be conducted again with stronger hardware.
It is to be noted that the implementation still performs decently 
well, given the relatively weak hardware it was tested on.

A comparison with existing systems and game engines is difficult. ATLOD is 
developed specifically for terrain rendering, whilst game engines
contain other components and often perform various tasks in the background, 
which hinders an accurate performance comparison.

\chapter{Conclusion}
\section{Potential Improvements}
The GeoMipMapping implementation has some room for improvement:
\begin{itemize}
      \item The view-frustum culling can be implemented more efficiently with a quadtree. The main problem with quadtree-based view-frustum culling 
            is that in order to support non-square terrains, special care needs to be taken for the quadtree size.
            A simple solution would be to define the quadtree to have a side length of the next power of two 
            larger than $\max \{terrainWidth, terrainHeight\}$ and to mark nodes as \texttt{null} in quadrants where there is no terrain.
      \item The performance can be further increased with \textit{instanced rendering}. This would reduce the number of draw calls dramatically.
      \item The idea that a $(0,0,1,0)$ border permutation is simply a $(1,0,0,0)$ border permutation with a rotation of $-\pi/2$ can be applied
     to further reduce GPU memory usage. This could be achieved by allocating only the indices for the border permutations $(0,0,0,0),(1,0,0,0),(1,1,0,0),(1,1,1,0)$ and $(1,1,1,1)$
            and then by simply rotating the flat mesh in the vertex shader, in addition to translating it.
      \item Another potential improvement is to extend the implemented algorithm with vertex morphing in order to reduce the popping artifacts.
\end{itemize}

\section{Outlook for the Bachelor Thesis}
There are several possible project ideas for the bachelor thesis which build upon this project and the topics behind it:
\begin{itemize}
      \item Integration of a terrain LOD system in a game engine (e.g. Godot) or in a scene-graph library (e.g. SLProject).
      \item Development of a flight simulator, where the user can control the aircraft/camera using gestures.
      \item Implementation of a streaming/paging-based terrain LOD algorithm, where multiple terrain instances are dynamically loaded and offloaded depending on the position.
            This would allow for (theoretically) infinite terrains. 
      \item Implementation and benchmarking of additional terrain LOD algorithms. Some interesting and relevant algorithms that could be added 
      are GPU-based Geometry Clipmaps, CDLOD and Concurrent Binary Trees.
\end{itemize}