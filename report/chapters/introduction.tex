\chapter{Introduction}
In 3D computer graphics, rendering is the central task.
Many practical applications of 3D computer graphics make use of terrains, 
such as flight simulators, open-world video games, and Geographic Information Systems (GIS) \cite[p.~185]{lodfor3dgraphics}.
At the same time, rendering terrains, which are large and constantly visible, is computationally expensive 
and optimizations are necessary in order to ensure adequate performance.

One area which offers potential for optimizations is the \textit{level of detail (LOD)}.
The concept of LOD is based on the intuitive idea that the farther away an object is, the fewer details are going to be visible to the human eye.
Over the last three decades, numerous algorithms and approaches have been published 
for the problem of efficient terrain rendering.

\section{Goals of this Project}
The primary goal of this project is the survey and evaluation
of terrain rendering algorithms.
First, the basics of terrain rendering are studied and 
an overview of the state of the art of terrain rendering is 
composed. Afterwards, a demo terrain renderer is developed,
using the ideas from one or more of the evaluated algorithms.

This project is restricted to rendering terrains from heightmaps without streaming/paging.

\section{Intended Readership}
The reader is assumed to be familiar with the basics of computer graphics, C++ and OpenGL.

\section{Notation and Terminology}
\subsection{Mathematical Notation}
This report uses the following mathematical notation:
\begin{itemize}
      \item The coordinate system is a right-handed coordinate system with $y$ as the up-direction, unless explicitly stated otherwise.
      \item $\mathbb{N}$ denotes the set of natural numbers, $\mathbb{R}$ is the set of real numbers, $\mathbb{R}^n$ is the set of real numbers in $n$ dimensions.
      \item $\mathbf{p} = (p_x, p_y, p_z)$ denotes a point in $\mathbb{R}^3$.
      \item $\mathbf{v} = (v_x, v_y, v_z)$ denotes a vector in $\mathbb{R}^3$.
\end{itemize}
If at any point the report contains mathematical notation which is
not described here, 
mathematical notation as commonly found in computer graphics is used.

\subsection{The Term ``LOD Level''}
The definition of what ``LOD level 0'' and what ``LOD level $l$'' ($l = $ maximum LOD) mean is different from paper to paper.
Normally, LOD systems use 0 for the highest resolu`ion and $l$ for the lowest resolution.
In this report, the opposite (and slightly more intuitive) approach is followed: $l$ denotes the highest resolution, and 0 denotes the lowest resolution.

\section{Outline of the Report}
This report is structured as follows:
\begin{itemize}
  \item Chapter 2 introduces the reader to the basics of terrain rendering. The topics covered include 
        terrain data representation, common optimizations and potential problems during rendering.
  \item Chapter 3 gives an overview of the state of the art of terrain rendering.
        Various algorithms and their central ideas are presented in a high-level manner.
        Afterwards, some examples of real-world systems using terrain LOD algorithms are given.
  \item Chapter 4 describes the demo terrain renderer (named ATLOD) which was developed.
        The basic features and the implementation details of the algorithms are presented.
  \item Chapter 5 Lists a few real-world examples where terrain LOD is being used, such as 
        game engines or geographic information systems.
        The algorithms are presented in a high-level manner.
  \item Chapter 6 describes ATLOD, the demo application for terrain rendering implemented as part of this project.
        An overview of the functionalities is given and the main approaches and design decisions are discussed.
  \item Chapter 7 TODO
\end{itemize}


