\chapter{Introduction}
In the field of 3D computer graphics, rendering is one of the central tasks.
Many practical applications of 3D computer graphics make use of terrains, 
such as flight simulators, open-world video games, and Geographic Information Systems (GIS) \cite[p.~185]{lodfor3dgraphics}.
At the same time, rendering large and constantly visible objects, such as the terrain, is computationally expensive 
and optimizations are necessary in order to avoid performance deficiencies.

One area which offers potential for optimizations is the \textit{level of detail (LOD)} of objects.
The concept of LOD is based on the idea that the farther away an object is, the fewer details are going to be visible to the human eye.

The problem of rendering terrains spawned numerous algorithms and approaches specifically
for this purpose. 

\section{Goals of this Project}
%The goals of this project can be formulated as follows:
The main goal of this project is to gain an overview over the field of 
terrain LOD, in theory as well as in practice. 
This includes introducing the basics of terrain programming,
analyzing and comparing existing approaches and algorithms, and
researching what approaches are used in the real world.
For this purpose, a terrain demo application (named \textit{ATLOD}) is developed
in C++ and OpenGL.

% \begin{itemize}
%   \item Introduction to terrain LOD: 
% \end{itemize}
% Goals: Introduction to LOD and terrain LOD, give overview of algorithms and implementations, 
%        overview of applications in game engines and other, demo application for trying out LOD algorithms, 
%        performance comparison 

\section{Structure of the Report}
This report is structured as follows:
\begin{itemize}
  \item Chapter 2 gives an overview of work that has already been conducted in the area of terrain LOD
        and relevant literature for this project.
  \item Chapter 3 Lists a few real-world examples where terrain LOD is being used, such as 
        game engines or geographic information systems.
  \item Chapter 4 introduces the reader to various approaches for terrain LOD, beginning with some basic background
        information on terrain modelling and LOD approaches in general. Afterwards, a selection of algorithms
        are presented in detail, including ROAM, GeoMipMapping, Quadtrees, Geometry Clipmaps, CDLOD. The algorithms are presented in chronological order of their publication date.
  \item Chapter 5 describes ATLOD, the demo application for terrain rendering implemented as part of this project.
        An overview of the functionalities is given and the main approaches and design decisions are discussed.
  \item Chapter 6 TODO
  \item Chapter 7 TODO 
\end{itemize}


